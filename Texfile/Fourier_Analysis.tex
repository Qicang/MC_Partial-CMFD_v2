\section{Fourier Analysis}
\subsection{Analysis Overview}
\begin{frame}
	\frametitle{Fourier analysis is a technique that estimates the convergence properties of a method by looking at how error modes decay in a model problem*}
	\begin{itemize}
		\item Procedure for Fourier analysis:
		\begin{itemize}
		    \item Define 1-D homogeneous problem with reflective boundary conditions
			\item Define a ``Fourier ansatz'' consisting of equations of the form
				\begin{equation}
					\phi^{(l)}(x) = \phi_{0} + \epsilon\sum_{\omega_j} \theta(\w)^l  e^{i \w_j x} \,,
				\end{equation}
        		where $\phi_{0}$ is the exact solution and the second term is a ``small'' error of frequency $\w_j$
			\item Substitute temporary solutions of the equations  which define your method by Fourier ansatz , drop terms of $O(\eps^2)$ or smaller, and perform some algebra to obtain an eigenvalue problem for the decay rate $\theta(\w)$
			\item The spectral radius of the method is then given by $\rho = \max\limits_{\w_j}|\theta|$, $\rho<1$ -- the scheme is stable.
		\end{itemize}%
	\end{itemize}
\end{frame}
\subsection{Model Problem}
%%%%%%%%%%%%%%%%%%%%%%%%%%%%%%%%%%%%%%%%%%%%%%%%%%%%%%%%%%%%%%%%%%%%%%%%%%%%%%%%%%%%%%%%%%%%%%%%%%%%%%%%%%%
\begin{frame}
\frametitle{Model Specification}
\begin{itemize}
\item Cross section is assumed to be linearly dependent with the localized flux~\footfullcite{kastenberg1969stability}:
\begin{equation}
\Sigma_{i}^{\left(n\right)}\left(x\right) = \Sigma_{i,0} + \Sigma_{i,1} \left(\phi^{\left( n\right)}\left(x\right) - \Phi_0 \right)\label{eqn:updatexs}
\end{equation}
\vspace{-1em}
\begin{itemize}
    \item In general the relation between cross section and the flux is not this simple. However, when the solution is close to the true solution, this approximation is reasonable.
    \vspace{-0.5em}
    \item Former research~\footfullcite{Kochunas2017FourierSections} shows this assumption can make Fourier analysis tractable and reveal the stability of the Picard scheme in neutron transport calculation with feedback. 
    \item A proxy multiphysics application which maintains some representative behavior while allowing for a spatially flat solution
\end{itemize}

\end{itemize}
\vfill
\end{frame}

%%%%%%%%%%%%%%%%%%%%%%%%%%%%%%%%%%%%%%%%%%%%%%%%%%%%%%%%%%%%%%%%%%%%%%%%%%%%%%%%%%%%%%%%%%%%%%%%%%%%%%%%%
\begin{frame}{Model Iteration Scheme}

\begin{algorithm}[H]
\begin{algorithmic}[1]
\STATE{Initialize $\phi(x)$ and $J(x)$}
\FOR{$n=0$ to $N$}
\vspace{-0.2em}
\STATE{Update the Cross Section by \refeqn{updatexs}}
\vspace{-0.5em}
\STATE{Calculate the CMFD coefficients and nonlinear coupling term $\hat{D}$}
\vspace{-0.5em}
\STATE{Use the CMFD coefficients to form $M$ and $F$}
\vspace{-0.5em}
\FOR{$l=0$ to $L$}
\vspace{-0.2em}
\STATE{Solve $\l[ M - \lambda_sF \r] \bphi^{(l+1)} = \l[ \lambda^{(l)} - \lambda_s \r] F  \bphi^{(l)}$}
\vspace{-0.5em}
\ENDFOR
\vspace{-0.5em}
\STATE{Update the source of transport equation with coarse mesh solutions}
\vspace{-0.5em}
\STATE{Perform transport sweep}
\vspace{-0.5em}
\ENDFOR
\end{algorithmic}
\caption{Model Iteration Scheme}
\label{alg:seq}
\end{algorithm}
\end{frame}    

%%%%%%%%%%%%%%%%%%%%%%%%%%%%%%%%%%%%%%%%%%%%%%%%%%%%%%%%%%%%%%%%%%%%%%%%%%%%%%%%%%%%%%%%%%%%%%%%%%%%%%%%%%
 %ZZZZ emphasize importance of this analysis


