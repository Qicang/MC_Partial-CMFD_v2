\section{Conclusions}

%%%%%%%%%%%%%%%%%%%%%%%%%%%%%%%%%%%%%%%%%%%%%%%%%%%%%%%%%%%%%%%%%%%%%%%%%%%%%%%%%%%%%%%%%%%%%%%
\begin{frame}{Conclusions }
\begin{enumerate}
    \item The Fourier analysis of the iteration scheme accelerated by partially convergent NDA is performed. 
    \item The results support the observations that partial convergence can help to stabilize the multiphysics simulations, and near-optimal partial convergence exists though it is problem dependent.

    \item The partial convergence is shown to be equivalent to more well understood relaxation factor, and the relaxation factor induced by partially converging the lower-order problem is mainly imposed on the flat Fourier modes.

    \item The formula to determine the near-optimal partial convergence is put forward, provide people with intermediate measure to mitigate the stability issuse confronted in multiphysics simulation. 
\end{enumerate}
\end{frame}

%%%%%%%%%%%%%%%%%%%%%%%%%%%%%%%%%%%%%%%%%%%%%%%%%%%%%%%%%%%%%%%%%%%%%%%%%%%%%%%%%%%%%%%%%%%%%%%%%%%%%%%%%%%%%%%%%%%%%%%%%%%%%%%%%
%%%%%%%%%%%%%%%%%%%%%%%%%%%%%%%%%%%%%%%%%%%%%%%%%%%%%%%%%%%%%%%%%%%%%%%%%%%%%%%%%%%%%%%%%%%%%%%%%%%%%%%%%%%%%%%%%%%%%%%%%%%%%%%%%
\begin{frame}{Mext Step}
\begin{enumerate}
    \item Test the prediction from the Fourier analysis in realistic problem.
    \vspace{2em}
    \item Modify the iteration scheme in VERA-CS by using near-optimal partial convergence.
    \vspace{2em}
\end{enumerate}
 \end{frame}
%%%%%%%%%%%%%%%%%%

